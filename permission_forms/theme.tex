% 4900 configs

\usepackage[margin=2cm]{geometry}

% document
\usepackage{fancyheadings, verbatim}
\pagestyle{fancyplain}

% fonts
\usepackage[T1]{fontenc}
\usepackage{baskervillef}
\usepackage[varqu,varl,var0]{inconsolata}
\usepackage[scale=.95,type1]{cabin}
\usepackage[baskerville,vvarbb]{newtxmath}
\usepackage[cal=boondoxo]{mathalfa}

% list
\usepackage{enumitem}

% tables
\usepackage{cellspace, tabularx, multirow, xcolor}      % <---

\setlength\cellspacetoplimit{10pt}     % <---
\setlength\cellspacebottomlimit{10pt}  % <---
\setlength{\parindent}{0em}   % noindent by default

\renewcommand\tabularxcolumn[1]{m{#1}}
\newcolumntype{C}{>{\centering\arraybackslash}X}
\newcolumntype{L}[1]{>{\raggedright\arraybackslash}p{#1}}
\newcolumntype{E}[1]{>{\centering\arraybackslash}p{#1}}
\newcolumntype{R}[1]{>{\raggedleft\arraybackslash}p{#1}}
\addparagraphcolumntypes{C}          % <---
\newcommand\mcx[1]{\multicolumn{2}{>{\hsize=\dimexpr2\hsize+2\tabcolsep\centering}X|}{#1}} % <---


% forms
% this is needed for forms and links within the text
\usepackage[pdfborderstyle={/S/U/W 1}]{hyperref}  % attempt setting universal borders

\hypersetup{
  pdfauthor   = {chuang@sci.brooklyn.cuny.edu},
  pdfkeywords = {Brooklyn College},
  menubordercolor=0 0 0,
  menucolor=0 0 0,
  allbordercolors=0 0 0,
  pdflinkmargin=-1.0pt,
  colorlinks=false,
  pdfborderstyle={/S/U/W 1}
} 
\def\Fld@menubordercolor{0 0 0} % border color black
\def\Fld@height{0.4cm} %



\makeatletter

\def\hlinewd#1{ %
  \noalign{\ifnum0=`}\fi\hrule \@height #1 %
  \futurelet\reserved@a\@xhline
}

% https://tex.stackexchange.com/questions/179019/frame-box-around-chapter-section-headings
\newcommand\makeSecHead[4][\fbox]{%
    \@namedef{#2}{\@ifnextchar*{\@nameuse{#2@i}}{\@nameuse{#2@ii}}}
    %
    \expandafter\def\csname#2@i\endcsname*##1{\par\vspace{#4}\noindent
       #1{\parbox{\dimexpr\textwidth-2\fboxsep-2\fboxrule}{%
         \normalfont\normalsize#3\makebox[0pt][l]{}~##1}}\par\vspace{#4}}%
    %
    \expandafter\def\csname#2@ii\endcsname{\@ifnextchar[{\@nameuse{#2@iii}}{\@nameuse{#2@iv}}}%
    %
    \expandafter\def\csname#2@iii\endcsname[##1]##2{\par\vspace{#4}\noindent
      #1{\parbox{\dimexpr\textwidth-2\fboxsep-2\fboxrule}{%
        \refstepcounter{#2}\normalfont\normalsize#3\makebox[14pt][l]{\@nameuse{the#2}}~##2}}%
        \addcontentsline{toc}{#2}{\@nameuse{the#2}~##1}\par\vspace{#4}}%
    %
    \expandafter\def\csname#2@iv\endcsname##1{\par\vspace{#4}\noindent
     #1{\parbox{\dimexpr\textwidth-2\fboxsep-2\fboxrule}{%
       \refstepcounter{#2}\normalfont\normalsize#3\makebox[14pt][l]{\@nameuse{the#2}}~##1}}%
       \addcontentsline{toc}{#2}{\@nameuse{the#2}~##1}\par\vspace{#4}}%
  }

% \makeSecHead[\colorbox{gray!30}]{chapter}{\Large\bfseries}{24pt}
\makeSecHead[\colorbox{gray!30}]{section}{\Large\bfseries}{10pt}
\makeSecHead{subsection}{\large\bfseries}{12pt}
\makeSecHead{subsubsection}{\large\bfseries}{10pt}

\makeatother



    \renewcommand*{\DefaultHeightofText}{3ex}
    \renewcommand*{\DefaultWidthofText}{3cm}
    \renewcommand*{\DefaultHeightofChoiceMenu}{3ex}
    \def\DefaultHeightofCheckBox{0.8\baselineskip}
    \def\DefaultWidthofCheckBox{\baselineskip}
    \def\DefaultOptionsofRadio{print}

\newcommand{\makeStudentBlock}[1]{%
    \section*{Your Information}
    
    \begin{tabularx}{\linewidth}{XX}
    % \baselineskip=20pt
     \TextField[name=student_fname,width=6.2cm,charsize=12pt,bordercolor={0 0 0},borderstyle=U]{\textbf{First Name:}}  &
     \TextField[name=student_lname,width=6.2cm,charsize=12pt,bordercolor=0 0 0,borderstyle=U]{\textbf{Last Name:}} \\[5pt]
     \TextField[name=student_emplid,width=5.6cm,charsize=12pt,borderstyle=U,bordercolor=0 0 0]{\textbf{CUNYFirst ID:}} &
     \TextField[name=student_email,width=7.1cm,charsize=12pt,bordercolor=0 0 0,borderstyle=U]{\textbf{Email:}} \\[5pt]
     \TextField[name=student_semester,width=7cm,charsize=10pt,bordercolor=0 0 0,borderstyle=U,value=#1]{\textbf{Semester:}} &
     \ChoiceMenu[combo,name=student_section,width=4.5cm,bordercolor=0.5 0.5 0.5]{Section: }{VC1A, VC1B, VC1C} \\
    \end{tabularx}%
}